In evolutionary game theory, success-driven migration promotes the emergence and stabilization of cooperation. However, all existing models assume that occupied sites cannot be contested: migration occurs exclusively toward empty locations. In reality, individuals routinely attempt to appropriate resources from incumbents -- a behavior spanning kleptoparasitism in nature to property crime, corruption, and territorial invasion in human societies. Here we introduce the {\it social rule violation game}, which extends success-driven migration by allowing individuals to expel incumbents from occupied sites with probability $s$, reflecting the limits of institutional enforcement. Using Monte Carlo simulations on a two-dimensional lattice, we show that the interplay between cooperation, migration, and social rule violation produces three key phenomena. First, a phase transition in cooperation level occurs at a critical violation probability $s^*$ that depends on population density $d$ and migration range $M$. Second, a tipping point emerges: the fate of cooperation is determined by a switch in expected payoff structure occurring at $\sim\!10^5$ Monte Carlo steps -- long before cooperation visibly collapses at $\sim\!10^6$ steps. In the sustainable regime, expected payoffs from different strategies decouple, indicating that small cooperative clusters manage violations locally. In the collapse regime, payoffs remain coupled, signaling systemic fragility. Third, the system exhibits hysteresis: once the tipping point is crossed, only drastic reductions in violation probability ($\geq 75\%$) restore cooperation, whereas early, modest interventions ($\sim\!25\%$) suffice if applied before the regime switch. These findings provide a game-theoretic mechanism for understanding when norm violations cascade versus remain contained, with implications for the enforcement of property rights and the design of intervention policies.
