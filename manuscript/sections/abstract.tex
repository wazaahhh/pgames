According to John Locke (1689) ``The reason why men enter into society, is the preservation of their property [...] to limit the power, and moderate the dominion, of every part and member of the society." \cite{locke2014second} This quote exhibits the tension between the right to private property and the necessity to build and maintain cooperative ties for the mere goal of property enforcement. The underpinnings of this apparent paradox can be unveiled by investigating cooperation with individuals playing the {\it prisoner's dilemma} and endowed with {\it success-driven} migration, as well as an additional {\it property violation} rule. We find that cooperative societies subjected to property violation may either strive, or contrary, completely collapse following invasion of defectors. The faith of cooperative societies is determined early on by a regime switch, after which successful societies exhibit strong spatio-temporal decoupling, with the establishment of decentralized organizations that tackle locally property violation. If this decoupling does not occur, long range spatio-temporal dependences pose a serious long-term threat for the subsistence of society as a whole. The {\it social planner} may prevent such a collapse by substantially increasing private property enforcement in due time.