According to John Locke (1689) ``The reason why men enter into society, is the preservation of their property [...] to limit the power, and moderate the dominion, of every part and member of the society.
" \cite{locke2014second}: As long as property violations and enduring costs remain marginal, people can 
engage into trustworthy contractual business relationships, which in turn bring individual and collective wealth \cite{}. Despite the importance of property rights and its implication \cite{}, it remains unclear by which mechanisms property right violations undermine cooperative societies in which individuals can choose their strategies in a prisoner's dilemma game with neighbors (cooperate or defect), and can migrate either to empty locations, or by expelling other individuals from their location. While little property violation may be beneficial for cooperation (in particular for small migration ranges), beyond a threshold is not sustainable: The more property violation the more cooperation is likely to collapse (this threshold is controlled by population density and migration range cooperation). We find that cooperation may be maintained only if the payoff from {\it prisoner's dilemma update} remains smaller than the payoff resulting from {\it success-driven migration} or from {\it property violation}. In that case, payoffs from available strategies (prisoner's dilemma update, success-driven migration, property violation) get decoupled and reach stationarity, ensuring long-term cooperation, while strongly coupled strategies lead to the collapse of cooperation. When cooperation is sustained, migration ranges for success-driven migration and property violation remain rather small (regardless of the migration range), showing that mitigation of property violators is best achieved {\it locally}. If a population is already organized against property violators, it can in fact cope with up to 20\% more property violation than the maximum level at which it achieved enduring organization.