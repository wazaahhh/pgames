According to John Locke (1689) ``The reason why men enter into society, is the preservation of their property [...] to limit the power, and moderate the dominion, of every part and member of the society.
" \cite{locke2014second}. In many ways, the level of trust people place in a society is often measured by its ability to defend their individual {\it property} rights \cite{}. The decision to join and keep faith into a society is thus a social dilemma, controlled by the enforcement of individual rights. Here, we report the sudden phase-transition from cooperation to defection, in a world where property violations are possible when individuals imitate superior prisoner's dilemma strategies and show success-driven migration with the possibility to expel other individuals from their location. In our model, individuals are unrelated, and do not inherit behavioral traits.  They defect, cooperate and migrate (including by expelling neighbors in the migration range) selfishly when their expected pay-off is increased, and they do not know how often they will interact with another individual in their neighborhood. The threshold and the nature of the phase-transition between sustained cooperation and invasion of defector are controlled by the migration range and the population density. Our results suggest that societies can manage high levels of property violation, provided that individuals expelled by property violators find a {\it good enough} relocation. This relocation is made easy with large migration ranges. In the latter conditions, a minority of cooperative individuals can survive by forming clusters in a world surrounded by defectors.\\
