\section*{Discussion}

The {\it social rule violation game} reveals that cooperation in spatial public goods games exhibits a tipping point, where the fate of society is determined by local mechanisms that unfold long before the large-scale consequences -- cooperation collapse or stabilization -- become apparent. At the coarse-grained level, the tipping point and the switch in expected payoff order may remain unnoticed, as cooperation continues to increase even after the structural conditions for collapse have been set. This finding has broad implications for understanding how social norms, property rights, and institutional enforcement interact to sustain or undermine cooperative societies.

\noindent{\bf Robust-yet-Fragile Cooperation.} Our results are reminiscent of systems that evolve into ``robust-yet-fragile'' structures through optimization \cite{carlson2000highly,carlson2002complexity}. Cooperators use the migration rule to optimize their survival against social rule violations, forming clusters that are locally robust. Yet at the same time, when violations are sufficiently prevalent, cooperators aggregate into increasingly massive clusters that organize their own fragility and, ultimately, their collapse. This mechanism parallels cascading failures in complex networks, where highly connected components are robust to random perturbations but vulnerable to targeted attacks \cite{gao2016universal,dsouza2023robustness}. In our model, the ``targeted attack'' is the invasion of compact cooperator clusters by defectors through social rule violations -- a mechanism that becomes increasingly effective as cluster size grows.

\noindent{\bf Tipping Points and Sensitive Intervention.} The {\it social rule violation game} exhibits a tipping point that occurs well before the large-scale effects leading to collapse actually unfold. This mechanism is reminiscent of climate tipping points and other critical transitions in complex systems \cite{scheffer2009early,lenton2011early}, where long delays separate the activation of hardly noticeable mechanisms from the actual collapse. More recently, the concept of social tipping processes -- rapid qualitative shifts in social systems driven by self-amplifying feedback mechanisms -- has been formalized as a framework for understanding how small perturbations can trigger systemic transformations \cite{winkelmann2022social,otto2020social}. Our finding that the decoupling of expected payoffs signals an impending regime switch resonates with the search for early-warning signals of critical transitions \cite{scheffer2018quantifying}. Importantly, our hysteresis results demonstrate that intervention timing is critical: early, modest reductions in violation probability $s$ suffice to stabilize cooperation, while late interventions require drastic corrections -- a pattern consistent with the notion of sensitive intervention points in complex systems \cite{farmer2019sensitive}.

\noindent{\bf Broken Windows, Norm Violations, and Emergent Compliance.} A striking emergent property of our model is that only defectors commit social rule violations, even though no rule prohibits cooperators from doing so. This self-organized compliance arises because cooperators embedded in cooperative clusters have no payoff incentive to expel neighbors. The result connects to the ``broken windows'' thesis \cite{wilson1982broken,keizer2008spreading}, which posits that signs of disorder trigger further norm violations, causing disorder to spread. However, recent empirical work has produced mixed evidence for this thesis: replication attempts of the original field experiments have yielded ambiguous results, and the effects of disorder appear to depend critically on local social capital and the cost of violations \cite{keuschnigg2015disorder,lanfear2020broken}. Our model provides a theoretical mechanism that may help explain this empirical ambiguity: norm violations cascade and destabilize cooperation {\it only} when they exceed a threshold that disrupts the local cluster structure of cooperators. Below that threshold, cooperative communities can absorb violations locally through the decoupling mechanism, which is consistent with the finding that disorder effects are weaker in high-social-capital neighborhoods \cite{keuschnigg2015disorder}. Evidence that high crime rates encourage migration away from central cities \cite{sampson1986evidence} further supports the spatial dynamics our model captures.

\noindent{\bf Property Rights and Institutional Enforcement.} Our model abstracts the enforcement of property rights into a single parameter $s$, reflecting the aggregate probability that a social rule violation succeeds. This abstraction encompasses a range of enforcement mechanisms -- from informal social norms to formal institutions such as policing and judicial systems. Recent evolutionary models have shown that specialized third-party enforcement can emerge endogenously when enforcers maintain mutual accountability \cite{powers2023enforcement}, and that collective property rights are a prerequisite for sustainable resource governance \cite{hauser2024cultural}. In this context, the parameter $s$ can be understood as the residual violation probability after all enforcement mechanisms have been applied. Our results suggest that even modest enforcement failures -- small increases in $s$ -- can trigger catastrophic cooperation collapse once they push the system past the tipping point, underscoring the importance of maintaining robust institutional enforcement capacity.

\noindent{\bf Limitations and Future Directions.} Our model makes several simplifying assumptions. First, we assume a uniform probability of social rule violation across the grid, whereas crime and norm violations are known to develop in spatial hotspots. Heterogeneous enforcement landscapes could alter the phase transition dynamics significantly. Second, all individuals share the same migration range $M$, whereas real populations exhibit substantial heterogeneity in mobility. Third, our model does not include costly punishment mechanisms \cite{fehr2002altruistic,boyd2003evolution} or social exclusion strategies \cite{sasaki2013evolution,szolnoki2017second}, which could interact with the violation mechanism in non-trivial ways. Future work could explore whether cooperative clusters function as autonomous sub-communities that tackle violations at a local scale, connecting to work on cultural group selection \cite{richerson2016cultural} and the hierarchical organization of human social groups. Additionally, decomposing the parameter $s$ into components reflecting different enforcement institutions could yield richer policy implications.

\noindent{\bf Broader Implications.} The {\it social rule violation game} occupies a unique position at the intersection of evolutionary game theory, social tipping point dynamics, and institutional enforcement of property rights. Our findings have implications for policies on immigration, labor mobility, public discourse, and innovation -- any domain where the capacity to protect private resources from unilateral appropriation affects the provision of public goods. The tipping point mechanism we identify suggests that societies may appear resilient to increasing norm violations long after the structural conditions for collapse have been set, and that effective intervention requires detecting early-warning signals of regime change -- a challenge shared with climate policy \cite{lenton2011early,farmer2019sensitive} and the governance of complex socio-technical systems more broadly \cite{winkelmann2022social}.
