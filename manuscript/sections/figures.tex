

\begin{figure}[h!]
\begin{center}
\centerline{\includegraphics[width=14cm]{../figures2/phase_transition_d05.eps}}
\caption{{\bf a.} Final cooperation level after $N = 200$ iterations (i.e., $200 \times d \times L^2$ Monte Carlo Steps) as a function of social rules violation prevalence $s$ for population density $d = 0.5$. The larger the Moore neighborhood $M$, the more cooperation is promoted for small social rules violations ($s < 0.005$) and the more cooperation is resilient against increasing prevalence of violation, declining linearly ($c \approx 1 - 2.2\cdot s$) up to a phase transition ($s^* \approx 0.03$ for $M=3$, and $s^* \approx 0.065$ for $M=11$; larger migration ranges provide diminishing additional benefit). The phase transition is not sharp: simulations starting from identical initial conditions may end with thriving or collapsing cooperation. Yet the probability of cooperation collapse increases as social rules violations increase. {\bf b.} Cooperation as a function of population density: higher density is detrimental to cooperation under social rules violations. The colored areas show the phase transition regions for $M= \{ 1,2,5,7,11\}$. The dashed lines show the lower limit below which cooperation thrives with certainty. The dot-dashed lines show the upper limit beyond which collapse occurs systematically. The solid lines show the boundary of equal probability of cooperation or collapse.}
\label{fig:phase_transition}
\end{center}
\end{figure}


\begin{figure}[h!]
\begin{center}
\centerline{\includegraphics[width=17cm]{../figures2/viz_transition.eps}}
\caption{Starting from the same initial grid ($t=0$) and with the same parameters (grid density $d=0.5$, migration range $M=5$, and property violation $s=0.04$), we observe two different outcomes. {\bf a.} Cooperation is sustained. Compact clusters of cooperators maintain a low concentration of defectors at any place of the grid, and when density of defectors increases in some clusters, these clusters manage to merge with neighboring clusters of cooperators (brown and blue overlay ellipses). An almost surely stable state is reached. {\bf b.} When clusters of defectors get established (yellow, magenta, and orange overlay ellipses), they bring a critical mass to invade cooperators, even though cooperators may migrate to concentrate further into larger clusters. Crucially, at the phase transition (here $s=0.04$), it is sufficient that a single cluster of defectors gets formed for all cooperation to collapse. Cooperation may even increase for a while before collapsing (see {\bf b1}, {\bf b2}, and {\bf b3} in which cooperation increases while the seeds of collapse have been planted early on).}
\label{fig:viz}
\end{center}
\end{figure}


\begin{figure}[h]
\begin{center}
\centerline{\includegraphics[width=15cm]{../figures2/tseries_transition_3.eps}}
\caption{Evolution of expected payoff from player actions -- success-driven migration (green), property violation (blue), or strategy update (red) -- for the same initial conditions ($d=0.5$, $M=5$, $s = 0.037$; colored areas show the 25\%--75\% percentile ranges). Expected payoffs exhibit very similar dynamics at first ($t< 10^5$ iterations) between thriving cooperation ({\bf a}) and collapse ({\bf b}). At $t \approx 10^5$ iterations, structural changes occur that decide the outcome: although cooperation continues to increase until $t \approx 5\times 10^5$, in the non-sustainable case ({\bf b}) strategy update (red) yields the highest expected payoff, while in the sustained cooperation case ({\bf a}), strategy update never provides the highest expected payoff. In the latter case, a subtle crossing occurs at the transition point $t \approx 10^5$, where migration transitions from the action with highest payoff to a lower payoff, while property violation becomes the action with highest expected payoff. Insets {\bf 1} and {\bf 2} show the correlograms between expected payoffs after the tipping point: in the sustainable case ({\bf a}), cross-dependence between actions is very low, indicating strong decoupling. In the collapsing scenario, cross-dependence remains high, with increased migration causing more property violation (green correlogram) and more strategy updates (red correlogram). In the unsustainable scenario, cooperation still increases long after the tipping point (until $t \approx 5\times10^{5}$) and then abruptly collapses. Panels {\bf c} and {\bf d} show expected mobility distance from success-driven migration (green), property violation (blue), and forced move (cyan). In the collapsing scenario ({\bf d}), expected mobility reaches a high plateau ($\log_{10} U_{\text{mobility}} \approx 2.5$), while for thriving societies ({\bf c}), expected mobility is lower ($1.5 < \log_{10} U_{\text{mobility}} < 2$), close to that of property violators ($\log_{10} U_{\text{expel}} \approx 1.5$). The dynamics presented for $d=0.5$ and $M=5$ are representative of phase transitions occurring for any game with $M>1$ and density $0.2 \leqslant d \leqslant 0.8$.}
\label{fig:tseries}
\end{center}
\end{figure}


\begin{figure}[h]
\begin{center}
\centerline{\includegraphics[width=15cm]{../figures2/adaptation.png}}
\caption{Hysteresis and intervention timing. Starting from a limit case ($s = 4.2\%$, $d = 0.5$, $M = 5$) where cooperation is known to collapse (black line), corrective actions reduce the violation probability $s$ at the times indicated by vertical bars. Panels {\bf a}, {\bf b}, {\bf c}, {\bf d} show cooperation trajectories for violation reductions of $10\%, 25\%, 50\%, 75\%$ respectively. Successful trajectories (high cooperation achieved) are shown in cyan; failed trajectories (cooperation collapse) are shown in magenta. The x-axis is presented in logarithmic scale. The earlier the reduction is achieved, the more likely cooperation will stabilize at high levels ($c > 0.8$). However, small reduction ($10\%$, panel {\bf a}) hardly guarantees survival; medium reduction ($25\%$, panel {\bf b}) must be applied while cooperation is still increasing; high reduction ($50\%$, panel {\bf c}) is necessary to escape the edge of collapse; and only drastic reduction ($\geq 75\%$, panel {\bf d}) ensures restoration at any intervention time, provided cooperators still remain \cite{helbing2009outbreak}. Eliminating violations entirely ($s=0$) stabilizes cooperation but removes the beneficial mobility noise induced by small levels of violation (not shown).}
\label{fig:adaptation}
\end{center}
\end{figure}
