\section*{Conclusion}
The {\it property game} exhibits a tipping point, which occurs way before the large-scale effects leading to collapse actually unfold. This mechanism is reminiscent of other global problems, such as climate change, with long delay between the almost unnoticeable mechanism activation leading to collapse, and the actual edge of collapse. Our results show that in the case of private property, the mechanism is reversible, provided that authorities can act very pro-actively at the (little) cost of little correction action, or only with drastic property crime reduction (at high costs) if the problem is tackled after it gets noticed (reduction of cooperation).\footnote{I feel frustrated in the end because I think the message by John Locke was that people would start to cooperate (and form a society as this cooperation goes beyond a one-to-one cooperation) in order to keep private property violators at bay. However, here nothing shows that private property triggers some cooperation at the exception of very small amount of property violation. In future work, we can see the problem in 2 non-exclusive ways: The first solution is a little hand-wavy and considers that cooperation actually increases with a little property violation. Actually, one can ``boot" cooperation with ``only" 30\% cooperators (randomly scattered) at $t=0$ ($d=0.5,M=5$). With more trials, it may be possible to lower this threshold. Starting with small clusters (e.g. like a family) may also help lower the threshold. The second solution is more realistic but also more tricky technically. It consist in considering a cooperative cluster as ``an autonomous society" or at least a autonomous sub-component of an entire society, which may be the scale at which property violation is actually tackled in the decoupled (striving) regime. This ties back to hierarchies in social group sizes  (c.f., Dunbar and others), as well as the notion of private property in family (c.f. The origin of the family, private property and the state \cite{engels1978origin}?).We could explore this path, but (i) better visualization is needed to get more hints of what's going on, and (ii) one must develop a way to account for the size of these clusters and/or what is special about them. In my mind both ideas are not mutually exclusive: the first is more related to the organization process starting from random organization and the second is more how stability is established and ensured on the long term. We could imagine a way to boot cooperation in adverse conditions (which include property violation) and get cooperation going once clusters are established.}