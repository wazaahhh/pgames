\section*{Model}

Our {\it social rule violation} study is carried out for the prisoner's dilemma game (PD), which is used to model selfish behavior of individuals when cooperation is risky and defection is tempting, but where the payoff of defection on both sides $P$ (punishment) is inferior to the payoff of mutual cooperation $R$ (reward) \cite{axelrod1981evolution,nowak2006five}. A player may defect with payoff $T$ (temptation), which results in the sucker's payoff $S$ for the cooperating individual. The inequalities $T > R > P > S$ and $2R > T + S$ define the classical prisoner's dilemma. In this game, it is more profitable to defect, no matter what strategy the other individual selects. Thus, rational individuals are expected to defect when they meet once. Yet, as is well-known \cite{nowak2006five}, cooperation can be supported by repeated interactions \cite{axelrod1981evolution}, by intergroup competition with or without altruistic punishment \cite{traulsen2006evolution,fehr2002altruistic,boyd2003evolution}, and by self-organization of cooperators in spatial clusters \cite{nowak1992evolutionary,szabo2002phase,hauert2004spatial}. 
In such spatial evolutionary games, the level of cooperation in two-dimensional spatial games is further enhanced by disordered environments \cite{vainstein2001disordered} and by diffusive mobility \cite{vainstein2007does}. Strategy mutations, random relocations, and other sources of stochasticity can significantly challenge the formation and survival of cooperative clusters \cite{perc2017statistical}. Similarly, success-driven migration allows individuals to leave unfavorable neighborhoods and seek more favorable cooperative clusters \cite{vainstein2007does,jiang2010role,yang2010role,funk2019directed}, supporting cooperative clusters against the destructive effects of noise and preventing defector invasion in a large area of payoff parameters \cite{helbing2009outbreak}.\\

The {\it social rules violation} game extends the success-driven migration game by allowing migrations to any location within the migration range, including the possibility to expel players from locations with higher payoff. We assume a density $d$ of individuals scattered on a two-dimensional square grid, with periodic boundary conditions and $L\times L$ sites, which are either empty or occupied by one individual. Population density $d$ is set between $0.2$ and $0.8$.\footnote{For $d \rightarrow 0$, the population is not dense enough to warrant PD interactions, and for $d \rightarrow 1$, there are not enough free sites to offer success-driven migration opportunities.} Individuals are updated asynchronously, in a random sequential order, and each individual gets updated on average $N$ times (in total, the number of Monte Carlo Steps $MCS = N \times d \times L^2$, with $L = 200$). At each step, the randomly selected individual performs simultaneous interactions with the $4$ direct neighbors and compares the overall payoff with that of these neighbors. If one neighbor has a better payoff, the individual updates her PD strategy with that of her best-performing neighbor. (Performing the same simulations with 8 neighbors does not significantly change the picture.) In the absence of PD strategy update noise ($r=0$) and random mutation ($q=0$), an individual will necessarily copy a superior PD strategy found in her neighborhood.\\

The migration step is performed before the imitation step \cite{helbing2009outbreak}: an individual explores the expected payoffs for {\it all} sites in the Moore neighborhood $(2M + 1) \times (2M + 1)$ of range $M$ (see Figure SI \ref{fig:migration_diagram}). If the fictitious payoff on the target location is higher than in the current location, the individual moves to the site with the highest payoff. There is no migration noise, and thus migration probability $m=1$. If the target site is empty, then {\it success-driven} migration occurs. On the contrary, if the site with highest payoff is already occupied by another individual, the {\it social rule violation} migration occurs: with probability $s$, the focal player takes the site of the incumbent individual, who in turn is expelled to the best empty site in her own Moore neighborhood. With probability $1-s$ however, {\it social rule violation} does not succeed and the focal individual migrates to the best empty site within the migration range. In our model, the probability of social rule violation is set as a parameter, representing the limits of institutional enforcement (e.g., a government policing along with a judicial system). Yet, attempts to violate social rules occur only when local conditions are met, i.e., when the opportunity to do so brings additional payoff. In other words, institutional enforcement occurs {\it only} in last resort, with a probability of success $1-s$, and when implicit enforcement of social norms (i.e., deterrence through maintenance of cooperation and adequate spatial configurations) has failed. (See SI Section \ref{stepbystep} for a step-by-step description of decision rules.)\\

An emergent property of the model is that only defectors commit social rule violations, even though nothing in the game design forbids cooperators from doing so. This arises because a cooperator embedded in a cooperative cluster already occupies a high-payoff site and has no incentive to expel a neighbor: the highest-payoff sites in a cooperator's Moore neighborhood are typically already occupied by fellow cooperators, and the cooperator's current site is already among the best available. By contrast, a defector surrounded by cooperators has strong incentive to move to a cooperator-rich site, and will attempt to expel an incumbent if that site offers the highest payoff. This emergent separation between strategy (cooperate/defect) and behavior (violate/comply) is a notable feature of the model.
