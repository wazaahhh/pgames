\section*{Introduction}

In a number of public goods games, migration has been found to promote the emergence, evolution and stabilization of cooperation \cite{vainstein2007does,helbing2009outbreak,jiang2010role,yang2010role}. A mature body of work now frames these spatial cooperation phenomena through the lens of statistical physics, where Monte Carlo simulations and the theory of phase transitions have proven essential for understanding how collective cooperative behavior emerges from individual interactions \cite{perc2017statistical,perc2016phase}. More recently, directed and adaptive migration mechanisms have been shown to shape cooperation through spatial pattern formation: cooperator aggregation promotes heterogeneous clustering and enhances public goods production, while defector dispersal has homogenizing effects that inhibit cooperation \cite{funk2019directed,chen2025sensitivity,wang2021directional}. All such models imply that resources are exclusive and rival private goods with no possibility to challenge private property rights. In other words, an incumbent individual cannot be expelled from her site, as success-driven migration always occurs towards empty sites. However, in a number of real-life situations, resources and social situations may be challenged: attempts of unilateral moves (e.g., migrations) to appropriate resources or situation rent from incumbents is a widespread behavior in nature and society. \\

In 1859, Darwin famously described how the bronze cuckoo and other brood parasites lay eggs in the nest of a stranger host bird, destroy incumbent eggs, and in case of cuckoo's eggs destruction by the host, they would attack the nest to compel host birds \cite{darwin1859origin}. Other forms of parasitism include aggressive mimicry \cite{wickler1965mimicry} or kleptoparasitism: spiders steal or feed on prey captured by other spiders \cite{coyle1991observations}. In human societies, unilateral moves aimed at appropriating rival resources occur commonly at local \cite{wilson1982broken}, urban \cite{kleemans1999social}, national and regional scales \cite{weiner1992security}, as well as in information realms, such as for intellectual property \cite{hughes1988philosophy,green2002plagiarism} and privacy \cite{warren1890right,acquisti2013privacy}. \\

Whether exogenous \cite{kinman1966migration} or endogenous \cite{bursik1988social}, such unilateral moves may challenge the stability of societies \cite{keizer2008spreading}. Exogenous unilateral moves may be warlike, or they may follow civil migrations, which generally bring long-term social, cultural, and technological innovation, yet at the cost of short-term social tensions and resistance from portions of incumbent populations who perceive immigration as a threat against their private goods, including their culture or religion \cite{park1928human}. Unilateral moves may also be endogenous, when individuals challenge the rules of society for the sake of dominion. Such endogenous invasions may be either directed at other individuals, or indirectly, by undermining the institutions in charge of enforcing social rules \cite{schafer1974political,johnsongovernment}. The endogenous challenge stems from the tradeoff between the partial alienation of freedom associated with the implicit \cite{elster1989social,kandori1992social,cialdini1998} and explicit \cite{giddens1984constitution,scott2013institutions} rules of societies \cite{ault1979development,sened1997political}. According to John Locke, societies exist for that mere purpose: {\it ``The reason why men enter into society, is the preservation of their property [...] to limit the power, and moderate the dominion, of every part and member of the society.''} \cite{locke2014second}.\\

On the one hand, challenging social rules provides a variety of rewards associated with immediate gratification for the violator, such as stealing an object instead of accumulating capital to buy it \cite{gottfredson1990general}, or a business securing unfair competitive advantage through corruption \cite{ades1999rents}. On the other hand, societies devote resources to protect against the occurrence of unilateral moves: a local neighborhood may fence itself against thieves in a gated community \cite{helsley1999gated,wilson2000exploration}, while a country may build border protections to guard against undesired migrations \cite{jones2012border}. In the realm of information, intellectual property is protected against unauthorized access through protective copyright regimes and licensing, or by software means \cite{burk2005legal,bechtold2003present}. And people's integrity shall be protected against the prejudice of physical violence \cite{miller1993victim,crawford2016crime} and privacy violations \cite{warren1890right,acquisti2013privacy}.\\

The challenge for a society -- and perhaps its mere reason for existence \cite{locke2014second} -- is precisely to set the conditions for its stability in a way that generates resilience in face of unilateral moves, associated with social rules violations and the ability by individuals to monopolize rival and excludable private resources from incumbents. Indeed, both theoretical arguments \cite{demsetz1964exchange,ostrom1990governing} and recent empirical evidence \cite{hauser2024cultural} suggest that the capacity to establish and enforce property rights -- whether individual or collective -- is a prerequisite for the emergence of cooperation and the sustainable governance of shared resources. Third-party enforcement of property rights has been shown to emerge endogenously in evolutionary models when enforcers maintain mutual accountability \cite{powers2023enforcement}, echoing the institutional enforcement mechanism we study here.\\

Population density and migration range are critical parameters in spatial public goods games. The percolation threshold determines an optimal population density for public cooperation \cite{wang2012percolation}, while the critical mass of cooperators needed to sustain cooperation depends on group size and spatial structure \cite{szolnoki2010impact}. In our model, the interplay between density $d$, migration range $M$, and the probability of social rule violation $s$ determines whether cooperation can be sustained or collapses through a phase transition.

Below, we consider the {\it social rule violation game}, a public goods game with individuals playing a simple prisoner's dilemma \cite{axelrod1981evolution,von2007theory} with spatial mobility \cite{nowak1992evolutionary}. With some probability, individuals can perform unilateral moves toward already occupied sites and expel incumbents. Unlike social exclusion mechanisms studied in public goods games, where cooperators may exclude free-riders from group benefits \cite{sasaki2013evolution,liu2020exclusion,wang2021bilateral}, our model introduces {\it involuntary spatial displacement}: an agent can forcibly take over another's site, reflecting the rival and excludable nature of private property rights enforced by social rules, and {\it in fine} their importance for the provision of public goods. The probability to succeed in {\it social rule violation} influences the level of cooperation, and we find that little rule violation helps cement cooperative communities, capable of handling violations locally. However, with increasingly prevalent rule-violating activity, cooperators aggregate and form larger clusters against defectors. These massive clusters of cooperators turn out to be fragile against invasion by defectors, and they ultimately collapse.
