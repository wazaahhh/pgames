\section*{Introduction}
The protection of private property counts among the oldest \cite{benson1989enforcement} and the most pervasive social norms and legal provisions across societies \cite{sened1997political}. Likewise, numbers of living systems have developed strategies to protect against foreign invaders as well as insider threats (e.g. the immune system), in order to ensure exclusive consumptions of resources and maintain homeostasis among constituting components of organisms \cite{}. In human systems, John Locke posited the very existence of societies by individuals seeking protection for their private property \cite{locke2014second}. The protection of private property may be ensured by fostering cooperation and trust through social norms, emerging in endogenous fashions among individuals, hence reducing free-riding among individuals \cite{}. Property enforcement may also be brought by exogenous rules elaborated by institutions representing a collective social endeavor (Efficient institutions for the private enforcement of law \cite{friedman1984efficient}).\\


{\bf[sideline thought :]}For example, in capitalist economies, labor unions bring employees together in order to balance the power of employers, and to ensure that privileges are fairly distributed among all workers \cite{}; International organizations and international treaties aim to reduce the preponderant influence of powerful states. Even with highly cooperative institutions based on sharing resources, such as the commons, rules are implemented to ensure that individual rights are respected \cite{ostrom1990governing}. \\

{\bf [some sideline citations:]}
{\it The origin of the family, private property and the state} \cite{engels1978origin}. {\it The exchange and enforcement of property rights} \cite{demsetz1964exchange}. {\it The boundaries of private property: how shall private property be divided: legal and constitutional challenges in such regulations} \cite{heller1999boundaries}\\

This study is motivated by the observation that the individual property rights are precious and thus regularly challenged in a number of ways, including attempts to overtake real-estate \cite{anderson1975evolution}, human rights \cite{}, privacy \cite{warren1890right} and digital privacy \cite{acquisti2013privacy}. Successful abuses of property rights make the violator better off and leave the incumbent abused individual in a less favorable situation. For example, in case of real-estate property violation, the incumbent may suffer some negative utility and in the worst case might be forced to find another location to settle. Unless they are themselves aimed at the systematic abuse of individual rights, governments are concerned by individual property right violations, as property crime may undermine trust and cooperation between individuals. In turn, trust and cooperation help keep defection and abuse at bay \cite{}, although it has been found that some level of free-riding actually help establish and maintain cooperation \cite{}. One contemporary instance of individual property violation is (online) identity fraud, which may undermine trust in online commerce, online administration and more generally exchange of information \cite{}.\\

Unlike other systems (e.g. markets?), which may converge to a social optimal stable equilibrium, fixing the ``right" level of private property enforcement (and hence an affordable level of property crime) through self-organized social norms or through institutions has remained a complicated problem: Because property law enforcement may become intrusive and thus undermine the trust in the law enforcement institution (think of government surveillance) \cite{johnsongovernment}, people desire the strict minimum amount property enforcement required to maintain a cooperative society . Also, going back to Locke, the very action of entering society represents by itself the alienation of a substantial portion of ``libre arbitre". Additionally, property crime enforcement by an institution is a costly business (find figures?), which shall consume as little as possible resources. There is thus a natural incentive to determine and to fix the maximum level of property violation (resp. the minimum level of private property enforcement) beyond (resp. below) which cooperation and trust are in danger of collapse.\\

Here, we question the nature of the maximum affordable level of property violation beyond which society is no longer sustainable, and considering that this optimal point shall not be found at once (i.e., {\it ex-ante} when a society is trying to get estalished), we shall investigate when and how much corrective actions shall be taken to restore high levels of trust and cooperation before it becomes too late.\\

For that purpose, we consider a public goods ``trust" game,\cite{} in which individuals play the prisoners' dilemma with their neighbors, and may move within a migration range to empty locations (i.e., success-driven migration) or on the contrary to occupied locations (i.e. property violation), forcing the incumbent individual to move to the next best empty site.\\

Recent work in game theory has shown that success-driven migration enhances and even elicits the outbreak of cooperation in the presence of combined imitation + mobility randomness \cite{helbing2009outbreak}, while most previous studies found that mobility can undermine cooperation by supporting defector invasion \cite{}. \\ %The conditions under which cooperation is maintained have extensively studied \cite{}, including with mobility \cite{} and randomness \cite{}. \\

Property violation forces to consider the effects of migration in an original fashion, where randomness stems from the probability by a player to take over the more favorable location of another individual. The property violation level reflects the expected success of property violation given all means deployed by society to prevent it as an equilibrium of forces between property violations capabilities on the one hand, and property enforcement on the other hand.\\

Property violation is also highly related to density and migration range {\bf [bring  more context here $\rightarrow$ Maybe in The evolution of property rights: a study of the American West \cite{anderson1975evolution};The private enforcement of law \cite{landes1974private}]}. Say something on density versus crime.\\


