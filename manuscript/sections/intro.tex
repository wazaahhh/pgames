\section*{Introduction}

{\bf [say something striking about mobility and property violation]}

{\bf [tie back to some empirical driven research on the topic, if available]}


We explore how violations of {\it individual rights}, such as private property, undermine cooperation in nature and society. Living organisms develop regulation and defense strategies to minimize aggression from their environment \cite{}, but also against inner threats such as cancer cells or parasites \cite{}, while primitive and advanced human societies have developed laws and judicial institutions to enforce private property \cite{benson1989enforcement,sened1997political}.\\

Below, we model cooperation in a game-theoretical way \cite{}, and integrate a model of individual {\it property} right violations. This is motivated by the observation that individuals are subjected to violations of individual rights, such as their private properties \cite{}, digital privacy \cite{}, or even their human rights \cite{}. These events make the violator better off, taking a part of the victims assets, while the latter is left in with less favorable options. In the case of a real-estate property violation, the incumbent is forced out of her property, and needs to find another empty location. \\

In our model, individuals consider a location in which they would be better off within a migration range. How favorable a new neighborhoods expected to be is determined by test interactions with individuals in that area (��neighborhood testing��). If they select an already occupied location, the incumbent is expelled with some probability, and forced to find another empty location within her own migration range.\\
 
So far, the role of migration has received little attention in game theory \cite{}, probably because it has been found that mobility can undermine cooperation by supporting defector invasion \cite{}. However, this primarily applies to cases, where individuals choose their new location in a random way. In the case of success-driven migration, cooperation has been found to strive and even to spontaneously emerge in the presence of randomness in actions taken by individuals \cite{helbing2009outbreak}.\\

As we will show, {\it property violations} undermine cooperation in surprising fashions, depending on the migration range and population density. Yet if the migration range is large enough, a minority of cooperative individuals can form dense clusters and resist to defectors.

 %We explore how violations of {\it individual rights}, such as private property, undermine cooperation in nature and society. Living organisms develop regulation and defense strategies to minimize aggression from their environment \cite{}, but also against inner threats such as cancer cells or parasites \cite{}, while primitive and advanced human societies have developed laws and judicial institutions to enforce private property \cite{benson1989enforcement,sened1997political}. \\
%
% For example, in capitalist economies, labor unions bring employees together in order to balance the power of employers, and to ensure that privileges are fairly distributed among all workers \cite{}; International organizations and international treaties aim to reduce the preponderant influence of powerful states. Even with highly cooperative institutions based on sharing resources, such as the commons, rules are implemented to ensure that individual rights are respected \cite{ostrom1990governing}. \\
%
%Private Property in Primitive Societies \cite{benson1989enforcement}. The exchange and enforcement of property rights \cite{demsetz1964exchange}. The political institution of private property \cite{sened1997political}. The boundaries of private property \cite{heller1999boundaries}. The evolution of property rights: a study of the American West \cite{anderson1975evolution} The private enforcement of law \cite{landes1974private}. Efficient institutions for the private enforcement of law \cite{friedman1984efficient}. The origin of the family, private property and the state \cite{engels1978origin}. The right to privacy \cite{warren1890right}.\\ 
%
%Nevertheless, individual rights are constantly challenged, and societies with badly implemented or unenforced law against right violations are bound to collapse, because collective trust is undermined (e.g., a corrupted state or city, {\bf [find a concrete example]}).$\rightarrow$ if there is too much crime in a society, trust is breached, and participants stop cooperating.\\
%
%Despite the importance of individual rights in society, little research has investigated under which conditions cooperation can be maintained or on the contrary collapses, when individual rights are violated . There is a fundamental tension between the need for laws in increasingly crowded environment, cooperation and the violation of private property. Cooperation can only be sustained if an agent can trust other agents, if she knows that other agents cannot take profit from her exclusive right on a private property. \\
%
%
%\subsection*{Modern Challenges of Private Property}
%- Say something on unstable societies / systems
%- Say something on robust yet fragile societies / systems
%- Collapsing systems and game theory.
%
%
%%The self-organization of cooperative behavior is often described by the prisoner's dilemma. When played as an evolutionary public good game with mobility on a grid, cooperators tend to clusters to protect against defectors. Here, we report the sudden collapse of cooperation when defectors can expel cooperators with some probability $s$. We find that cooperative systems can afford some level of property violation $s*$. This level $s*$ drops dramatically as free resources (empty sites available for migration) get scarce. Hence, optimizing cooperative systems (with a larger portion of available resources used) are (dramatically) more sensitive to property violation, suggesting that increasing resources devoted to property enforcement are required in crowded environments.
%
%%Since cooperation is very sensitive to even small values of $s$ in case of noise, can we determine a maximum value of $s$ beyond which cooperation does no longer emerge? $\rightarrow$ redo the emergence of cooperation experiment with various values of $s$. (\textcolor{red}{I don't think it makes sense to try this because the result is already somehow there: since there is this drop, we know at about which level of cooperators the phase transition occurs. It's very unlikely that cooperation can grow if it is below this critical level.})
% % \item Given that $s*$ varies as a function of density $\rho$, can we introduce a growth function, which involves progressive replication of agents (AB models), and we check the evolution of the right level of enforcement? \textcolor{red}{(next paper)}.
%%\end{itemize}
%
%%These resources are however subjected to rivalry and theft. In most human societies, examples of property violations include pickpockets and real-estate abuse, violations of intellectual property, online identity frauds and privacy violation. These private property violations undermine trust and cooperation for the production of collective value. Besides individual protection against private property violations, most human societies set a level of legal, executive and judicial enforcement, which is considered as sufficient to maintain cooperation. But what should be the ``right" level of private property right enforcement to maintain cooperation among individuals? Or conversely, how much private property violation a cooperative society can afford? And how is this level of enforcement is influenced by the migration range of agents, as well as their density ?\\
%
%The conditions under which cooperation is maintained have extensively studied \cite{}, including with mobility \cite{} and randomness \cite{}.
%
%
%\subsection*{Conclusion Introduction}
%So far, the importance of {\it private property} on cooperation has received little attention in game theory, because {\bf [...]}. As we will show, sufficient private property enforcement is an crucial to maintain cooperative societies. And the more optimized a cooperative, the even more private property enforcement is needed.
%
%Private property violation implies that people take advantage of other people's position. Here, we study how private property violations undermine cooperation.

%\subsection*{Implementation}
%
%For that, we propose a public good game, in which, agents can steal the ``site" of a another agent, with some probability $s$. 
%
%
%- importance of saturated versus resourceful environments (maybe to be put further down in the paper) $\rightarrow$ tragedy of the commons !
%
%Below, we model cooperation in a game-theoretical game  way,  and we integrate the influence of the violation of exclusive rights, as well as the level of resources available (to some extent, maybe rephrase as ``opportunities").
%
%This is motivated by the observation that individuals ``best-placed" (with higher pay-off) trigger envy and jealousy by others. $\rightarrow$ actually, there are clusters of cooperators that defectors aim to destroy {\bf intentionally} (versus by chance in previous models).
%
%To improve their situation, individuals are often willing to migrate to a more favorable place. What if this place is already occupied and/or has become favorable precisely because it is occupied (think of a field well labored for years, compared to one, which has been abandoned for a long time) ?
%
%So far, the role of private property has received little attention in game theory.
%
%{\bf [Insert a paragraph on cooperation/tragedy of the commons / private property and how they connect to the model]}
%
%how much ``property enforcement" (by whatever means) is needed ? 
%
%As we will show, cooperation can only be sustained for when high levels of property enforcement exist. We study the influence of property violation in the migration game.
