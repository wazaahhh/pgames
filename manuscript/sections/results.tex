\section*{Results}
While the prospect of private property drives cooperation,\footnote{I don't know how this assertion finds an operationalization in the model $\rightarrow$ maybe rephrase or even drop} cooperation is itself sensitive to property violation. While small quantities of property violation introduce some amount of randomness that enhances cooperation (see Figure \ref{fig:phase_transition}a), we focus on the limit amount of property violation beyond which cooperation transitions to defection.\footnote{as said above the overall rational is that private property enforcement is costly for society, and hence there is an incentive to find the maximum bearable level of private property.}\\

Our model operationalizes the action of property violation as a uniform random variable [$p(S \geqslant s) = 1$], with sure strategy update ($r=0$),\footnote{for strategy update we resort to the update method proposed by Helbing and Yu \cite{helbing2009outbreak}, which in our noiseless case is equivalent to the Fermi update with Fermi temperature equal to 1.} no random strategy reset ($q=0$) and no random migration to an empty site ($m=0$). And as players choose their best shot in the success-driven migration step, we can say that they are fully rational agents (in their migration range), which choice to expel another individual results only from odds of property enforcement.\footnote{These parameters settings ensure the simplest possible model that guaranties that the observed stylized facts are genuine to the property violation phenomenon.} \\

\subsection*{Population density, Migration Range and Cooperation Level} 
The outcome of the property violation game is nevertheless constrained by the initial conditions, such as population density and migration range: Increased population density reduces the opportunities to find an empty location, and thus, mechanically increases the odds of property violation. The migration range on the one hand increases the chances to find an empty sweet spot in case of success-driven  or forced migrations. On the other hand, it also increases opportunities for property violators. Migration influences the capability by defectors to penetrate clusters of cooperators. However, expelled cooperators can move far enough to make the cluster move away from the defector. Hence, larger migration ranges increase tolerance to property violation (see Figure \ref{fig:phase_transition}b), but their beneficial effects are limited for highly densely populations. On the contrary, populations with very short migration ranges -- here, embodied by unitary migration distance ($M=1$) -- can hardly stand any property violation. 

%Nevertheless, the intensity of property violation is determinant in the ability by defectors to break clusters of cooperators %(see Figure \ref{fig:viz} and online visualization tool).

\subsection*{Optimizing for Property Violation is Managing at the Edge of Collapse}
Property enforcement is costly, and thus, there is an incentive to limit enforcement to the strict minimum, which will ensure high cooperation (which is equivalent to decide how much property crime a society can stand). For large enough migration ranges ($M\geqslant 3$), a little property violation ($s < 1 \%$), brings the necessary randomness that reduces friction and enhances cooperation, similarly to previous results obtained with other types of noise involved in success-driven migration games \cite{helbing2009outbreak}. For larger property violation $s > 1\%$, cooperation decreases linearly according to $ c = 1 - 2.2\cdot s$, until a critical phase transition is reached (see Figure \ref{fig:phase_transition}a), which itself depends on the population density and the migration range as reported on Figure \ref{fig:phase_transition}b. There is no hard limit, but rather a transition from sure striving society with high cooperation to a sure collapse. The region in between is large compared to the absolute value of the upper limit: For instance, for $M=5$ and $d=0.5$, the phase transition spans from $s = 3.7\%$ to $s= 4.2\%$, with an equal chance to survive or collapse for property violation $s=4\%$. Hence, the ``danger zone" spans over more than 10\% of the limit value, and the decreasing probability of striving in the critical phase transition region, makes it very hard to determine whether the course of a society will end up positively or negatively.\footnote{We have verified that the outcome is independent from the initial grid organization, by running multiple simulations from the same initial grid.}

\subsection*{Decoupling for Sustainable Cooperation}
The sudden phase transition from high cooperation to collapse suggests a ``grain of sand" factor deciding whether a society operating at the limit level of property violation will strive or collapse.\footnote{Again, it's important to stress that it is natural for a society to operate close to the limit, since private property enforcement is costly, and even sometimes socially undesired.}\\

To better determine this ``grain of sand", we resort to simple of economics implied by the property game and opportunities evolve over time: To maximize their payoff, individuals are endowed with three basic rules, 

\begin{itemize}
  \item {\bf strategy update rule:} the player updates its prisoner's dilemma strategy (i.e., cooperate or defect), as a function of her neighbors strategies,
  \item {\bf success driven migration:} the player migrates to the best empty site within her migration area determined by the Moore distance $M$,
  \item {\bf property violation:} the player attempts to expel with probability $s$ another player from her location within their migration area determined by the Moore distance $M$.
\end{itemize}

The three strategies are opportunity based, deterministic for the first two (i.e., if players are presented with better options they will update their strategy or migrate) and random for property violation, reflecting the unsure nature of taking the property of someone else. While decisions are idiosyncratic at the individual level, at the aggregate level, they reveal the expected utility $U$ carried by each rule, which can be formulated as 

\begin{equation}
U_{rule} = p_{rule} \times u_{rule},
\label{U}
\end{equation}

with $p_{rule}$ the probability to execute one of the 3 {\it rules} defined above (i.e., $rule = \{$update, migration, expel$\}$, and $u_{rule}$ the payoff increase from executing a {\it rule}. For prisoner's dilemma strategy update and success-driven migration $p_{rule} = (1~|~u_{rule} > 0)$ (reflecting the deterministic decision to execute these rules) and for property violation $p_{rule} = (s~|~u_{rule} > 0)$, reflecting the unsure nature of attempting to expel another individual. For each rule, $U_{rule}(t)$ provides a direct measure -- at the aggregate level -- of opportunity associated with each rule and how it evolves. Similarly, one can define an expected mobility distance,

\begin{equation}
N_{rule} = p_{rule} \times n_{rule},
\end{equation}

where $p_{rule}$ is similar as in (\ref{U}) and $n_{rule}$ is the migration distance. $N_{rule}$ shows the expected travel distance for success-driven and property violation migration rules.\\

Starting from a population randomly scattered across the grid with an even number of cooperators and defectors, $U$ and $N$ undergo significant changes over time, with subtle yet determinant differences between a striving and a collapsing society (see Figure \ref{fig:tseries} for an illustration of a successful and a failed outcome with similar parameters $d=0.5$, $M=5$ and $s=0.037$; This particular case illustrates similar stylized facts obtained for different values of $d$, $M$ and $s$). In both the successful and the failed dynamics, {\it success-driven} migration provides the highest expected payoff increase ($U_{migration} > U_{update} > U_{expel}$)  in the early iterations $t < 1.5\times10^{5}$ (over $20\times10^6$ MCS iterations). After $t > 1.5\times10^{5}$, a significant change occurs: In striving societies (Figure \ref{fig:tseries}b) a switch of expected payoff occurs, which becomes : $U_{expell} > U_{update} > U_{migration}$, while in the failed society case (Figure \ref{fig:tseries}b), strategy update provides the highest expected increased payoff, and the expected increased payoff from success-driven and property violation migrations converge.\footnote{ nb: It is important to note that expected payoff is not necessarily increasing overall the entire population: players expelled from their site suffer a negative expected payoff increase (not shown), as well as neighbors around the focal player who changes prisoner's dilemma strategy to defect, or a defector migrating or performing a property violation.} At the time of the switch, we observe an inflection point of the cooperation level (purple dashed circle on Figure \ref{fig:tseries}b): After this point, cooperation still increases in the failed case, but at much lower pace,\footnote{Mostly driven by strategy updates, which provide the highest expected payoff increase? These updates are at first of the type : $D \rightarrow C$ and then, after the peak rather $C \rightarrow D$. The intuition is that, somewhen migration (whether success driven or property violation) does bring enough payoff increase, compared to strategy update, therefore something weird occurs which is that cooperation increases (more or larger clusters?), which in turn prepare for an invasion of defectors!} on the contrary to striving societies for which we observe a smooth continuous increase of cooperation, up to saturation and stabilization at a high level ($c > 0.8$). In the failed scenario illustrated in Figure \ref{fig:tseries}, the peak of cooperation is reached at  $t \approx 5\times10^5$, and collapse occurs after roughly$10^6$ iterations, nearly $850,000$ iterations after the switching point!\\

While it is hard to observe the detailed switching mechanisms occurring at the transition point,\footnote{This may be done in the future.} we find that the expected mobility (the probability to remains high in the failed scenario in comparison with the successful case (Figures \ref{fig:tseries}c and \ref{fig:tseries}d).\footnote{It is important to note that mobility by property violation and consecutively by individuals forced to move, is not nearly as large as the mobility induced by success-driven actions (c.f., Figures \ref{fig:tseries}b Inset 2).} Also, we find that in the successful scenario, a strong decoupling between expected payoff increase occurs (Figures \ref{fig:tseries}a inset 1), while for the collapsing scenario, dependence between expected payoff remain high and get even reinforced (Figures \ref{fig:tseries}b inset 2).\footnote{The correlogram before the switch also shows high dependence (not shown here), which is natural somehow, as the system gets organized {\bf [This correlogram (pre-switching) should be shown too, at least in the supplementary materials. In the failing scenario, it shows that property violation may influence ]}} Moreover, in the latter case, success-driven migration actions strongly drive private property violations (green correlogram) and strategy updates (red correlogram).\footnote{The correlation peak occurs for a negative lag (roughly 15-20 bins {\bf [I must check the size of a bin in terms of iterations]}) when considering the influence of private property violations and strategy updates on migration} Dependence between property violation and strategy update is also high (blue correlogram), but with no lag thus one cannot say whether $E \rightarrow U$ or $U \rightarrow E$. One can infer however that they are both influenced by success-driven migration (see red and green correlograms).\footnote{It is unclear however what makes success-driven migration the driver of other actions. Something special must occur before or at the switching point.}\\

Past the very first iterations where success-driven migration is the predominant way to increase payoff (in order to build and reach clusters of cooperators, regardless of whether the moving player is a cooperator or a defector), our findings draw two completely different stories leading to either collapse or well-established cooperative societies. In the latter scenario, expected payoff from success-driven migration rapidly decays to a level comparable to the expected payoff derived from strategy update and property violation. Additionally expected payoff increase from the three rules get highly independent: A spatio-temporal decoupling occurs and players no longer undergo long-range migration, and while property violation brings the highest expected payoff increase, it is tackled by some kind of locally established self-organization \footnote{(which remains to be explained/documented in further details)}. On the contrary, the collapse scenario is driven by long-range migrations,\footnote{We need perhaps more insights on the nature of these migrations, e.g., do they spread defectors or rather cooperators? Does this evolve over time?} which in turn trigger strategy updates and property violation.\footnote{These long-range spatio-temporal dependences look like systemic risk to me, which may appear in a seamless, subtle and hardly noticeable way!} \footnote{It would be interesting to check the dynamics with a residual level of property violation, e.g., $s=0.01$, and with a super high level of $s$, to see if these extreme cases magnify differences between success and failure.}

\subsection*{Enduring and Adaptive Societies Facing Property Violation}
One striking observation is that even though the system switches early on, and determines the final outcome, changes may remain unnoticed for a while, as cooperation increases for 4 times more iterations (since $t=0$) until collapse actually occurs.\footnote{In some simulations, the inflection point in the increase of cooperation is hardly noticed (e.g. Figure \ref{fig:adaptation})}\\

Thus, on the one hand it may take time to realize that a society is running towards the edge of collapse, and on the other hand, it may take time to take corrective action, such as taking actions to reduce private property violations.\footnote{As a result of Figure \ref{fig:phase_transition}b, one could also think of increasing migration ranges (which would require massive and long-term infrastructure investments) and decreasing population density, which concretely in our world would mostly require reducing population and would be ethically questionable. Here, we limit our study to private property violations as the main control parameter.}\\

Budgets allocated to property right enforcement may vary over time, in particular, a typical situation is that property crime is low and therefore there is a reduced sense of urgency and budgets might be reduced as a consequence. Also on the long term, institutions concerned with property enforcement may experience downturns of efficiency (e.g., through some kind of corruption).\\

Here, we ask two questions:  

\begin{enumerate}
  \item {\bf Resistance:} Given that a cooperative society is in the process of establishing \footnote{I am currently running simulations for this.} or has stabilized at a high level of cooperation (given a level of property violation), how much more property violation and for how long can this society stand a higher level of property violation?
  \item {\bf Adaptation:} Given that a society is on the path to collapse (as shown e.g., on Figure \ref{fig:tseries}b), how early and by how much property violation shall be reduced to ensure it will strive by reaching a stable high level of cooperation?
\end{enumerate}

\paragraph{Resistance:} We find that when a society has stabilized at a high cooperation level ($s=4\%$), increasing a property violation a little ($0.05 \leqslant s \leqslant 0.06$) increases the chance that cooperation will collapse suddenly yet on the long run ( iterations > $3 \times 10^5$ after change of property violation level). For higher levels of  property violation ($s>0.06$), the cooperation curves decreases in a more or less parabolic shape, which seems to be invariant in the limit of $s \rightarrow \infty$, and it looks like that the zero-cooperation level can be hit at earliest around $3 \times 10^5$ after the increase of property violation (see Figure \ref{fig:resistance}).

\paragraph{Adaptation:} The tipping point  ($t \approx 10^5$) occurs way before the edge of actual collapse ($t \approx 10^{5.7}$), and thus, a very natural question is whether it is possible to take corrective actions to prevent the collapse of cooperation once the tipping point has passed and has 
been detected. We find that unless property violation can be reduced drastically (e.g., 75\%, c.f., Figure \ref{fig:adaptation}d), action must be taken at latest when cooperation is at its maximum  to ensure resilience (see Figure \ref{fig:adaptation}c when property violation is halved). For more realistic property violation reduction (10\% to 25\%, c.f. Figures \ref{fig:adaptation}a and\ref{fig:adaptation}b), the policy maker must intervene as the cooperation still increases. We could of course imagine more complicated and progressive property violation mitigation schemes (which is actually more relevant for drastic actions to reduce of property violation that may take time to take effect), but the idea is that in order to avoid collapse, action must be taken {\bf before} the effects of property violation get actually widely noticed.\footnote{with more simulations, we could actually compute thoroughly the odds of survival given timing and effort deployed to reduce property violation, but I think what is presented here is enough to demonstrate that it is super important to avoid doing what politicians do -- or are forced to do by the ``real-politics" democratic mechanisms, which is taking action only when the problem get sufficiently noticeable by the population.}\\

