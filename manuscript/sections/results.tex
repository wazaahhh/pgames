\section*{Results}
Computer simulations of the {\it social rules violation game} show that a small amount of violation enhances cooperation by introducing disturbance similar to noise: cooperation peaks ($c \approx 0.98$) for $s \approx 0.005$ \cite{helbing2009outbreak}. As the prevalence of social rules violations increases, cooperation reduces roughly linearly as $c \approx 1 - 2.2\cdot s$ (see Figure \ref{fig:phase_transition}a). 

\subsection*{Phase Transition}
However, when the probability of social rules violations exceeds a threshold $s^*$, a sharp change of regime occurs: at the phase transition point, cooperation may thrive or collapse. As social rules violations become more prevalent, the probability of cooperation collapse increases (see Figure \ref{fig:phase_transition}a). The phase transition point $s^*$ varies positively with migration range $M$ and negatively with population density $d$. Densely populated environments are very sensitive to social rule violations, regardless of mobility (see Figure \ref{fig:phase_transition}b).\\

At the phase transition $s^*$, the collapse of cooperation is triggered by the formation of local clusters of defectors. These clusters may be neutralized by cooperators (see Figures \ref{fig:viz}a and \ref{fig:viz}b for an instance of neutralization by cluster merging, and Figures \ref{fig:viz}e, \ref{fig:viz}f and \ref{fig:viz}g for successful proliferation of social rules violations and defector invasion). When social rules violations spread, cooperators evade defectors and form larger clusters (Figure \ref{fig:viz}g). These larger clusters help maintain a high level of cooperation. However, as they become more compact, these clusters become increasingly fragile to defector invasion (through social rule violations) and finally collapse (Figure \ref{fig:viz}h). 

\subsection*{Tipping Point}
The decision to update strategy is bound to the expected payoff of playing a PD strategy (i.e., cooperation vs.\ defection), success-driven migration, or social rule violation (only the latter is governed by a random variable). The expected payoff $U$ is empirically measured as the sum of all payoffs from all individuals for a given strategy played within a time window (see SI Section \ref{exp_utility}). It therefore encompasses not only the payoffs themselves, but also the number of times the strategy is played. The evolution of the expected payoffs shows that initially success-driven migration yields the highest payoff $U_{\text{migration}}$, followed by strategy update $U_{\text{PD update}}$ (see Figure \ref{fig:tseries}). Social rules violation is the strategy with least expected payoff $U_{\text{violation}}$ (i.e., $U_{\text{violation}} \leqslant U_{\text{PD update}} \leqslant  U_{\text{migration}}$). This period corresponds to the formation of clusters (see Figures \ref{fig:viz}a and \ref{fig:viz}b). As clusters get formed, the expected payoffs of success-driven migration, strategy updates, and social rules violations decrease and converge, yet their order remains unchanged. Around iteration $MCS = 10^{5}$ (see SI Section \ref{tippingpoint}), two qualitatively different scenarios unfold, associated with a change of expected payoff order:

\begin{enumerate}
 \item {\bf Decoupling and local management of social rules violations:} When cooperators manage to settle in relatively small clusters that are robust against social rules violations, strategy changes become de-correlated (see Figure \ref{fig:tseries}a, inset 1). Further evidence is provided by the expected displacement from success-driven migrations (see SI Section \ref{exp_mobility}): the expected displacement within the Moore neighborhood drops over time (see Figure \ref{fig:tseries}c). Hence, with cluster formation, individuals move more locally. PD strategy updates yield higher expected payoff, and thus more strategy changes occur (see the increased volatility of strategy updates in Figure \ref{fig:tseries}a). Social rule violations yield the highest expected payoff ($U_{\text{migration}} \leqslant U_{\text{PD update}} \leqslant U_{\text{violation}}$). Yet, the organization in small clusters of cooperators is highly robust, and deters defectors and social rule violators, even though the expected payoff of violation is higher. This occurs because, with small well-organized clusters, defectors are few in number and gravitate around cooperation clusters. They are only occasionally tempted by social rule violations, as the spatial structure of clusters limits their opportunities to do so. Defectors remain divided and hence cannot create their own clusters.
 
\item {\bf Coupling and systemic fragility build-up:} The outcome leading to collapse shows that it is sufficient for one cluster of defectors to form (cf.\ Figure \ref{fig:viz}e) for defection to spread, forcing small clusters of cooperators to group into more massive clusters (Figure \ref{fig:viz}f), which in turn become fragile (Figure \ref{fig:viz}g) and disappear (Figure \ref{fig:viz}h). The systemic component of defector spread is exhibited by the high cross-correlations between expected payoffs [see Figure \ref{fig:tseries}b, inset 2, after the tipping point ($t^* \approx 10^5$)]. The expected payoff order switches and strategy update yields the highest expected payoff ($U_{\text{violation}} \leqslant U_{\text{migration}} \leqslant  U_{\text{PD update}}$). Yet, the cooperation level continues to increase (see Figure \ref{fig:tseries}b) past a hardly noticeable inflection point (dashed red circle). At some point cooperation stops increasing, the order of expected payoffs changes again ($U_{\text{migration}} \leqslant U_{\text{violation}} \leqslant  U_{\text{PD update}}$) with success-driven migration yielding less expected payoff as massive conversion to defection occurs. Finally, cooperation drops and collapses abruptly around $MCS = 10^6$ iterations (cf.\ Figure \ref{fig:tseries}b).
 \end{enumerate}

The phase transition and tipping point dynamics described above are robust across population densities $0.2 \leqslant d \leqslant 0.8$ and migration ranges $M > 1$. Higher density increases the sensitivity to violations (the critical $s^*$ decreases), because expelled individuals have fewer empty sites available for relocation. Larger migration ranges increase the critical $s^*$ and promote cooperation by enabling cooperators to form clusters more efficiently and to escape defectors over greater distances. However, beyond $M \approx 11$, additional migration range provides diminishing returns, as the local cluster dynamics that sustain cooperation operate at scales smaller than the Moore neighborhood.\\

In summary, the tipping point manifests as a regime switch in the expected payoff structure that occurs early in the simulation ($MCS \approx 10^5$), while cooperation continues to rise and the eventual collapse does not materialize until much later ($MCS \approx 10^6$). The key diagnostic is the cross-correlation structure between expected payoffs: decoupling signals sustainability, while persistent coupling signals impending collapse. This delay between the structural tipping point and the observable collapse is a defining feature of the model, and echoes the general phenomenon of ``critical slowing down'' observed in other complex systems near tipping points \cite{scheffer2009early}.

\subsection*{Hysteresis and Resilience}

A natural policy question arises: if the system has crossed the tipping point, can cooperation be restored by reducing the violation probability $s$, and if so, by how much and how quickly must intervention occur \cite{farmer2019sensitive}?

We test the hysteresis properties of the {\it social rules violation} game by taking a limit case with $s = 4.2\%$ ($d=0.5$, $M=5$), where cooperation is known to collapse after some time (see black thick line in Figure \ref{fig:adaptation}). From the tipping point ($t^* \approx 10^5$ iterations), we ask how long before collapse and by how much the probability of social rules violations must be reduced.\footnote{One may interpret this reduction as the result of political, legislative, or enforcement actions, which for simplicity translate into an aggregate budget allocated to reducing social rules violations.}

The earlier the reduction of social rules violations is achieved, the more likely cooperation will thrive and stabilize at high levels ($c > 0.8$). However, small violation reduction of the order of $10\%$ hardly guarantees that a cooperative society will thrive (see Figure \ref{fig:adaptation}a). Medium reduction of the order of $25\%$ must be implemented while cooperation is still increasing (see Figure \ref{fig:adaptation}b). High reduction ($50\%$) is necessary to exit the edge of collapse and stabilize society at high cooperation levels (see Figure \ref{fig:adaptation}c). Only drastic reduction ($\geq 75\%$) ensures restoration of cooperation at any point in time (see Figure \ref{fig:adaptation}d), as long as a sufficient number of cooperators remain \cite{helbing2009outbreak}. Eliminating violations entirely ($s = 0$) stabilizes cooperation but simultaneously removes the mobility noise -- induced by small levels of violation -- that helps cooperation thrive (not shown).

These results exhibit clear hysteresis: the violation probability required to restore cooperation from the collapsed regime is substantially lower than the critical $s^*$ at which the phase transition occurs. Moreover, the spatial structure of small cooperative clusters that characterizes the thriving regime is also restored when the intervention is sufficiently early and strong, as evidenced by the recovery of the decoupled expected payoff structure and local migration patterns.
