\section*{Results}
Computer simulations of the model presented above show that the probability of property violation $s^{*}$ beyond which cooperation cannot not survive is highly dependent on the migration range $M$ and the population density $d$. At time $t=0$, the simulation starts with an equiprobable number of cooperators and defectors scattered uniformly across occupied sites.\\ 

When there is no migration ($M=0$), and by definition no property violation ($s=0$), evolution occurs only through replication of direct neighbor strategies. Cooperators form clusters to prevent invasion by defectors (see Figure \ref{fig:configurations_t200}{\it A}). In absence of property violation, any migration range $M \geqslant 1$ unambiguously promotes cooperation (see for instance Figures \ref{fig:configurations_t200}{\it B} and \ref{fig:configurations_t200}{\it E}, and defectors rapidly disappear (the migration range has no effect on the speed at which cooperators invade the population. See Figure SI \ref{figSI}).\\

However, considering the evolution of cooperation when individuals have small migration range $M \leqslant 5$, even little property violation can completely destroy cooperation, past an abrupt critical (phase transition) point $s^{*}$. For instance, for a unit Moore's migration range, a $8\%$ probability property violation is sufficient to destroy cooperation (see Figure \ref{fig:configurations_t200}{\it D}, as well as Figure \ref{fig:configurations_t200}{\it G} for $M=5$). For $s < s^{*}$, a majority of cooperators ($c>0.5$), along with a minority of defectors, which cluster together for $M=1$ (see Figure \ref{fig:configurations_t200}{\it C}) or scatter around clusters of cooperators for $M \geqslant 5$ (see e.g., Figure \ref{fig:configurations_t200}{\it F}).\\

With larger Moore migration ranges ($M \geqslant 5$), cooperative populations can sustain much more property violation. For instance, for population density $d=0.5$, cooperators account for on average 80\% of the population after 200 iterations, with property violation as large as $s^{*}_{-} = 0.45$ with migration range $M=11$ (see Figure \ref{fig:configurations_t200_M11plus}{\it B}), as well as for $M=13$ and nearly half chance for property violation (see Figure \ref{fig:configurations_t200_M11plus}{\it F}).\\

Also for large migration ranges ($M \geqslant 5$), an intermediary state $s^{*}_{-} \leqslant s \leqslant s^{*}_{+}$ appears, in which cooperative populations can survive while on average in minority (see Figures \ref{fig:configurations_t200_M11plus}{\it C} and \ref{fig:configurations_t200_M11plus}{\it G}). For $s > s^{*}_{+}$, cooperative populations do not strive (see Figures \ref{fig:configurations_t200_M11plus}{\it D} and \ref{fig:configurations_t200_M11plus}{\it H}).\\

As individuals search their {\it best-shot} -- maximizing expected pay-off at the selected location (in their Moore's migration range $M$) -- they only try to expel another individual, with property violation probability $s$, if this {\it best-shot} location is already occupied. The effects on cooperation, are not only a function of property violation, but also of the migration range as well as the population density: The lower the density $d$ and the larger the migration range $M$, the more opportunities to find a {\it best-shot} location, which is empty. On the contrary, the higher the population density and the smaller the migration range, the more individuals must rely on property violation to make a successful move. As shown on Figure \ref{fig:heatmaps} for $M= \{ 5,7,11,18\}$, the level of property violation $s^{*}$ beyond which cooperation cannot survive, depends on both the migration range $M$ and the population density $s$. For large $M \geqslant  7$, the intermediary state $s^{*}_{-} \leqslant s \leqslant s^{*}_{+}$, where cooperative populations can survive while on average in minority, appears in green. We also find that for high population densities $d > 0.9$, cooperation cannot survive for $s > 0$; Even with no property violation $s=0$, there is a probability that cooperators cannot invade the whole population, in particular for large migration ranges (see SI Section \ref{SI:d09}).\\


\subsection*{Going Further}
To further understand the properties of the {\it property game} as a special {\it success-driven} migration game, we shall consider how individuals exploit the degrees of freedom offered to them, such as empty sites available when $d < 1$ and their capabilities to reach these sites according to their mobility $M > 0$, how they achieve even greater opportunities by expelling individuals from their location. We finally investigate how cooperation is affected by the relocation of expelled individuals. In our model, individuals are fully rational as they will imitate they imitate their neighbors and migrate with probability $1$, if the find a strategy and a location respectively, with higher payoff. Only the probability of property violation is uncertain and controlled by $0 \leqslant s \leqslant 1$. \\

{\bf We consider the actual mobility if individuals $m = \sqrt{|x^2 + y^2|} \leqslant M$ bounded by the migration range $M$, versus average distance of sites (weighted by their payoff and whether they are free or not). This has something to do with the structure of clusters $\rightarrow$ a proxy could be density of cooperators vs. defectors in the migration range}.

{\bf We shall also who uses property violation? defectors or cooperators?}

{\bf Note that the larger the migration range, the more likely to find a number of sites for which the payoff is the same. In that case, the migration site is chosen in the following way: randomly among empty sites, and if there is no empty site, randomly among non empty site with some probability of property violation $s$. Note that calculating payoff from 8 neighbors instead of 4 may help alleviate this intrinsic randomness.}

{\bf Note that if the migration range is large enough, agents can ``jump" from one cluster to another $\rightarrow$  think of hubs (technology, finance, politics)}
