

\section{Social Rule Violation Game Step-by-Step}
\label{stepbystep}

Here, we present a detailed explanation of the decision process involved in the social rule violation game, at each Monte Carlo Step: 

\begin{enumerate}
  \item {\bf Player selection:} At each Monte Carlo Step $MCS$, a site is selected ($MCS = N \times L^2$ with $N$ the number of iterations, and $L$ the side of the square grid). If the site is occupied, the {\bf focal individual} is selected. Each individual is expected to be selected $d \times N$ times with $d$ the grid density. 
  
  \item {\bf Exploration:} With probability $1 - m$, the focal individual strategically explores its own migration (Moore) neighborhood $(2M + 1) \times (2M + 1)$ of range $M$, searching for a site with better payoff given current prisoner's dilemma (PD) strategy, i.e., either cooperate or defect. To assess for sites with best payoff, the individual plays the prisoner's dilemma with her own strategy and with neighbors for each site within the migration neighborhood. For each site assessed, the ``virtual'' payoff is computed as the sum of outcomes from playing the prisoner's dilemma with all neighbors.
  
  \item {\bf Move to best empty location (success-driven migration):} If among the sites with highest payoff, some are empty, the individual moves to the closest empty site.
   
  \item {\bf Move to best occupied location (social rule violation):} If there is no empty site among those with highest payoff, the focal individual expels the target individual with probability $s$. The expelled target individual is forced to move to the closest empty location with highest payoff within her own migration range. The expelled individual may find a new empty site with either lower or higher payoff. For both the focal and the target individual, in case multiple sites with highest payoff are available, the closest one is automatically selected. If they are at the same distance, one site is randomly chosen among best sites with smallest migration distance.
  
  \item {\bf Move to better empty location (success-driven migration):} If there is no empty site among those with highest payoff and property violation did not occur with probability $1-s$, the focal individual moves to the closest empty location with higher -- yet not highest -- payoff.
  
  \item {\bf No migration:} If all empty sites in the migration range have a payoff worse than the incumbent payoff on the focal site, the individual does not move.
   
 \item {\bf PD strategy imitation:} Whether it has moved or not, the individual is allowed to update her PD strategy (cooperate or defect) by playing the prisoner's dilemma with her neighbors. If one neighbor's strategy leads to a better payoff, the focal individual imitates this strategy with probability $1-r$. With probability $r$, her strategy is reset: the individual cooperates with probability $q$ and defects with probability $1-q$. If the target individual is forced to move after a property violation, she is not allowed to update her strategy (because it is not her turn to play). In this study, we consider no imitation noise (i.e., $r=0$). Hence, individuals {\it systematically} copy a more successful PD strategy.
\end{enumerate}

\begin{figure}[h!]
\begin{center}
\centerline{\includegraphics[width=7cm]{../figures2/migration_diagram.eps}}
\caption{Migration diagram for focal player (defector on crossed site with $\text{payoff} = 1.3$). The best site in the migration range has highest payoff for the focal player (red-green site with $\text{payoff} = 1.3 \times 4 = 5.2$). The target site with highest payoff is occupied by a cooperative player and may be expelled with probability $s$ (orange arrow). In this case, the player on the target site is forced to move to the nearest empty site with $\text{payoff} = 2$ (black arrow pointing to light green site). However, with probability $1-s$, the focal player moves to the best empty site with $\text{payoff} = 2.6$ (blue arrow pointing to light green site).}
\label{fig:migration_diagram}
\end{center}
\end{figure}


\section{Simulation Parameters}

All simulations are performed on a two-dimensional square lattice with periodic boundary conditions and side length $L = 200$ (i.e., $40{,}000$ sites). Population density $d$ is set at initialization by randomly occupying a fraction $d$ of sites. At initialization, each occupied site is assigned a cooperator or defector strategy with equal probability ($c_0 = 0.5$). Each individual is updated on average $N = 200$ times over the course of a simulation, yielding a total of $MCS = N \times d \times L^2$ Monte Carlo Steps. The prisoner's dilemma payoff parameters are set to $T = 1.3$, $R = 1$, $P = 0.1$, $S = 0$, satisfying the conditions $T > R > P > S$ and $2R > T + S$. Migration probability is $m = 1$ (no random migration), imitation noise $r = 0$, and random mutation $q = 0$. The key parameters varied across simulations are:
\begin{itemize}
  \item Population density: $d \in \{0.2, 0.3, 0.4, 0.5, 0.6, 0.7, 0.8\}$
  \item Migration range (Moore neighborhood): $M \in \{1, 2, 3, 5, 7, 9, 11, 15, 20\}$
  \item Social rule violation probability: $s \in [0, 0.2]$
\end{itemize}
For each parameter combination, multiple independent simulations are run with different random seeds to capture the stochastic variability of outcomes, particularly near the phase transition.


\section{Emergent Compliance: Why Only Defectors Violate}
\label{behaviors}

An emergent property of the social rule violation game is that cooperators never commit social rule violations, even though the model imposes no such constraint. This arises from the payoff structure of the game in combination with the spatial organization of clusters.

Consider a cooperator occupying a site in a cooperative cluster. Her current payoff is high because she interacts primarily with other cooperators ($R = 1$ per cooperator neighbor). The sites with highest payoff in her Moore neighborhood are similarly embedded in cooperative clusters and are already occupied by cooperators. There are two cases: (i) the best site is empty -- in which case the cooperator migrates there via success-driven migration, not social rule violation; or (ii) the best site is occupied by another cooperator -- in which case expelling that cooperator would not improve the focal cooperator's payoff, as the target site's neighborhood structure is similar to the current one. Hence, cooperators have no payoff incentive to expel incumbents.

By contrast, a defector surrounded primarily by other defectors receives low payoffs ($P = 0.1$ per defector neighbor). Sites adjacent to cooperative clusters offer substantially higher payoffs (up to $T = 1.3$ per cooperator neighbor). These high-payoff sites are typically occupied by cooperators. The defector thus has strong incentive to expel the incumbent cooperator and take the high-payoff site. This asymmetry in incentive structure ensures that, in practice, only defectors attempt and commit social rule violations.


\section{Expected Utility of Playable Strategies}
\label{exp_utility}

To characterize the dynamics of the tipping point, we measure the expected utility of each playable strategy over time. For each Monte Carlo Step, the action taken by the focal individual is recorded along with the resulting payoff change. We define the expected utility $U_k(t)$ of strategy $k$ (where $k \in \{\text{migration}, \text{PD update}, \text{violation}\}$) within a time window $[t - \Delta t, t]$ as:
\[
U_k(t) = \sum_{i \in \text{events}(k, [t-\Delta t, t])} \Delta \pi_i
\]
where $\Delta \pi_i$ is the payoff change resulting from event $i$, and the sum runs over all events of type $k$ within the time window. This measure captures both the frequency with which a strategy is played and the average payoff it yields: a strategy that is played often but yields small payoff changes may have the same expected utility as one played rarely but yielding large changes.

The time window is set to $\Delta t = 1000$ Monte Carlo Steps. The expected utility is computed separately for success-driven migration, PD strategy updates (cooperate $\leftrightarrow$ defect), and social rule violations (including both successful expulsions and failed attempts). By tracking the ordering of $U_{\text{migration}}$, $U_{\text{PD update}}$, and $U_{\text{violation}}$ over time, we identify the tipping point as the moment when this ordering switches.


\section{Expected Mobility}
\label{exp_mobility}

Similarly to the expected utility, we measure the expected mobility to characterize how far individuals move over the course of the simulation. For each migration event (success-driven or violation-induced), we record the Euclidean distance $d_i$ traveled by the focal individual. The expected mobility $E_{\text{mob}}(t)$ within a time window $[t - \Delta t, t]$ is:
\[
E_{\text{mob}}(t) = \frac{1}{|\text{events}|} \sum_{i \in \text{events}([t-\Delta t, t])} d_i
\]

In the thriving regime, expected mobility decreases over time as cooperators settle into stable clusters and move only locally (within the immediate neighborhood). In the collapse regime, expected mobility remains high, reflecting the large-scale displacement of cooperators fleeing from invading defectors. The contrast between local (low-mobility) dynamics in the thriving regime and systemic (high-mobility) dynamics in the collapse regime is a key diagnostic of the tipping point.


\section{Detecting the Tipping Point}
\label{tippingpoint}

The tipping point is identified by monitoring the cross-correlation structure between the expected utilities of the three playable strategies. In the early phase of the simulation ($MCS < 10^5$), the expected utilities of migration, PD update, and violation are positively correlated and their ordering is consistent: $U_{\text{violation}} \leqslant U_{\text{PD update}} \leqslant U_{\text{migration}}$.

At approximately $MCS \approx 10^5$, the dynamics diverge:
\begin{itemize}
  \item In the {\bf thriving regime}, the cross-correlations between expected utilities drop sharply, indicating that the three strategies become {\it decoupled}. Migration, strategy update, and violation operate largely independently, reflecting the fact that small cooperative clusters manage violations locally without systemic repercussions. The expected utility ordering switches to $U_{\text{migration}} \leqslant U_{\text{PD update}} \leqslant U_{\text{violation}}$.
  \item In the {\bf collapse regime}, the cross-correlations remain high, indicating that the strategies remain {\it coupled}. A violation triggers migration, which triggers strategy updates, creating a positive feedback loop. The expected utility ordering switches to $U_{\text{violation}} \leqslant U_{\text{migration}} \leqslant U_{\text{PD update}}$, reflecting the dominance of mass strategy switching from cooperation to defection.
\end{itemize}

The decoupling/coupling diagnostic can be computed from the lag cross-correlation coefficients between $U_{\text{migration}}(t)$, $U_{\text{PD update}}(t)$, and $U_{\text{violation}}(t)$ evaluated after the regime switch (see Figure \ref{fig:tseries}, insets 1 and 2).


\section{Resilience and Hysteresis}
\label{resilience}

To test the hysteresis properties of the social rule violation game, we start from a limit case ($d = 0.5$, $M = 5$, $s = 4.2\%$) where cooperation is known to collapse after approximately $10^6$ Monte Carlo Steps. At various time points after the tipping point ($t^* \approx 10^5$ iterations), we reduce the violation probability $s$ by a fixed percentage ($10\%$, $25\%$, $50\%$, or $75\%$) and observe whether cooperation recovers. Results are shown in Figure \ref{fig:adaptation} (main text).


\clearpage


\subsection*{Parameter Interpretation}

The rationale for varying the population density stems from the intuition that densely populated areas are more resource-intensive (i.e., available resource per individual is more scarce), and thus property violation has more negative effects, such as making relocation more difficult for the expelled individual. We indeed found that beyond a given population density ($d > 0.8$), cooperation cannot survive in the presence of even the slightest level of property violation.

The large span of Moore migration range, from very local mobility ($M=1$) to levels close to full mobility on the grid ($M=24$ for a grid of $49\times49$), reflects large inequalities in mobility at local, regional, and global scales. The migration range should be viewed as relative to the world under scrutiny: for $M=1$ the individual can only move by a small fraction of the world size at each step; for $M=12$, the individual may move within half the world. Clusters may form locally, but since only one individual can occupy a grid site, density remains overall well-distributed. The migration range applies to all individuals, regardless of whether they migrate to an empty site or attempt to expel another player. There is no designated property violator in our model: all individuals may become property violators based on their payoff opportunity and the probability of overcoming enforcement.

The probability of property violation reflects the limits of enforcement, whether of legal, police, or military nature, or as a result of individuals' capacity to protect their own assets.


\subsection*{Migration and Property Games in Densely Populated Worlds ($d \geqslant 0.9$)}
\label{SI:d09}

In the success-driven migration game, a fully populated world ($d=1$) is only feasible when property violation exists ($s > 0$), since an individual willing to move must expel another individual. In the absence of property violation ($s=0$), the game only involves updating strategies with no migration. When $d=1$ and $s > 0$, we find that cooperators cannot thrive and disappear after fewer than 15 iterations.

In the limit of population density $d \rightarrow 1$, success-driven migration with property violation $s > 0$ leads to a systematic collapse of cooperation. However, for highly dense worlds (e.g., $d = 0.9$), even without property violation ($s = 0$), while cooperators can resist defectors, they may not be able to fully invade the grid and may become stuck at an intermediate level ($0.4 \leqslant c \leqslant 0.6$). Cooperators successfully thrive and invade nearly the entire grid with some probability, a phenomenon that appears to depend on both initial conditions and the stochastic dynamics of individual-level updates.

\begin{figure}[H]
\begin{center}
\centerline{\includegraphics[width=12cm]{../figures/configurations_d09_s0.eps}}
\caption{Evolution of cooperation for $d=0.9$, with $M = \{1,3,5,7,9,11\}$. Cooperation performs best for medium migration ranges $M = \{3,5\}$. For $M \geqslant 7$, cooperation tends to stabilize between 40\% and 55\%. For all $M$, cooperators counter defector invasion by creating small clusters, which grow up to a saturation point where these clusters can survive surrounded by defectors. In some cases ($M = \{3,5,9\}$), cooperators manage to break through the belt of defectors and achieve high cooperation levels.}
\label{fig:medium_d_migration}
\end{center}
\end{figure}

\begin{figure}[H]
\begin{center}
\centerline{\includegraphics[width=13cm]{../figures/configuration_d09_s0.eps}}
\caption{Effects of migration in densely populated grids ($d=0.9$). With small migration range $M=1$ (see {\bf B.}), clusters of cooperators form quickly, along with defectors around these clusters. The small migration range does not allow cooperators to jump over defectors. When the migration range is larger $M=5$ (see {\bf C.}), similar clusters of cooperators form at first ($t = 35$), but the larger migration range allows creating connected clusters ($t = 56$), which in turn help overcome most defectors ($t = 200$).}
\label{fig:high_d_migration}
\end{center}
\end{figure}


\clearpage

\subsection*{Migration and Property Games in Medium Populated Worlds ($0.4 \leqslant d \leqslant 0.6$)}

The medium-density regime ($0.4 \leqslant d \leqslant 0.6$) is the primary focus of the main text, as it exhibits the richest dynamics including the phase transition, tipping point, and hysteresis phenomena. In this regime, there are enough empty sites for success-driven migration to operate effectively, while the population is dense enough for meaningful PD interactions and cluster formation.

\begin{figure}[H]
\begin{center}
\centerline{\includegraphics[width=14cm]{../figures/TimeSeriesPhaseTransitions.eps}}
\caption{Evolution of cooperation for average grid density $0.40 \leqslant d < 0.60$ over $N=200$ iterations for migration ranges $M = \{ 1,3,5,7,9,11,13\}$ and for property violation values $s$ close to $s^*(M)$ (migration probability $m=1$). The presented time series illustrate how the property violation phase transition occurs as a function of migration range. Populations with small migration range $M=1$ can enhance cooperation after an initial drop, yet limited migration capabilities make populations very sensitive to property violation: for $s^{*} > 0.05$, defectors win quickly (see {\it A.}). For $M=5$, we observe two scenarios: either cooperative populations win ($s < s^{*} = 0.158$) or they disappear ($s>s^{*}$). Populations that cannot sustain a cooperation level above $0.5$ almost surely collapse to full defection (see {\it B.}). As migration range increases ($M = \{7,9\}$), an intermediary state appears in which populations sustain clusters of cooperation while defectors are in the majority (see {\it C.} and {\it D.}). For larger migration ranges ($M = \{11,13\}$), this intermediary state becomes more prominent (see {\it E.} and {\it F.}), with cooperative populations in the minority ($c \approx 0.45$ for $M=11$ and $c \approx 0.40$ for $M=13$), yet clustered in a world of defectors.}
\label{fig:tseries_phase_transition}
\end{center}
\end{figure}

